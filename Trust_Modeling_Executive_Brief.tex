\documentclass[11pt,a4paper]{article}
\usepackage[utf8]{inputenc}
\usepackage[margin=1in]{geometry}
\usepackage{amsmath}
\usepackage{amsfonts}
\usepackage{amssymb}
\usepackage{graphicx}
\usepackage{booktabs}
\usepackage{hyperref}
\usepackage{xcolor}
\usepackage{fancyhdr}
\usepackage{enumitem}
\usepackage{array}
\usepackage{multirow}

% Header and footer
\pagestyle{fancy}
\fancyhf{}
\fancyhead[L]{Trust Modeling in Stag Hunt Games}
\fancyhead[R]{\today}
\fancyfoot[C]{\thepage}

% Custom colors
\definecolor{darkblue}{RGB}{0,51,102}
\definecolor{lightblue}{RGB}{173,216,230}

\title{\textbf{\Large Trust Dynamics in Stag Hunt Games:\\A Computational Study of Bounded Rational Agents}}
\author{Mobin Zarreh \\ Arizona State University}
\date{\today}

\begin{document}

\maketitle

\section*{Executive Summary}

This study develops and validates computational models of trust formation in strategic interactions using the Stag Hunt game framework. We compare a novel dual-state agent model incorporating psychological biases against a Bayesian rational benchmark, analyzing how different partner strategies affect trust evolution and decision-making outcomes.

\section{Research Motivation and Objectives}

\subsection{Problem Statement}
Traditional game-theoretic models assume perfect rationality, but real-world agents exhibit psychological biases in trust formation. Understanding these dynamics is crucial for:
\begin{itemize}[nosep]
    \item Designing collaborative systems with human-AI interaction
    \item Predicting behavior in multi-agent environments
    \item Developing trust-aware algorithms for autonomous systems
\end{itemize}

\subsection{Research Questions}
\begin{enumerate}[nosep]
    \item How do psychological biases (loss aversion, surprise penalties) affect trust formation compared to rational Bayesian updating?
    \item Which partner strategies most effectively build or erode trust over time?
    \item What are the decision-making and payoff implications of different trust models?
\end{enumerate}

\section{Methodology}

\subsection{Game Framework: Stag Hunt}
We use the canonical Stag Hunt payoff matrix representing collaboration dilemmas:

\begin{center}
\begin{tabular}{c|c|c|}
\multicolumn{1}{c}{} & \multicolumn{1}{c}{\textbf{Cooperate}} & \multicolumn{1}{c}{\textbf{Defect}} \\
\cline{2-3}
\textbf{Cooperate} & (4,4) & (0,3) \\
\cline{2-3}
\textbf{Defect} & (3,0) & (2,2) \\
\cline{2-3}
\end{tabular}
\end{center}

The critical threshold for cooperation is $p^* = \frac{2}{3}$ (probability partner cooperates).

\subsection{Agent Models}

\subsubsection{Model 1: Bayesian Rational Agent}
Uses standard conjugate prior updating with Beta distribution:
\begin{align}
p_t &= \frac{\alpha_t}{\alpha_t + \beta_t} \\
\alpha_{t+1} &= \alpha_t + \text{cooperation}_t \\
\beta_{t+1} &= \beta_t + (1 - \text{cooperation}_t)
\end{align}

\subsubsection{Model 2: Dual-State Bounded Rational Agent}
Maintains separate signal ($x$) and trust ($t$) with asymmetric updating:

\textbf{Expected Probability:}
\begin{equation}
p_t = \frac{\varepsilon + x_t \cdot t_t}{2\varepsilon + t_t}
\end{equation}
where $\varepsilon = 0.5$ is the prior parameter.

\textbf{Trust Update:} Category-weighted with psychological biases:
\begin{equation}
\text{Consistency} = \frac{W_M}{W_M + \lambda W_B + \mu W_U + \epsilon}
\end{equation}

Where:
\begin{itemize}[nosep]
    \item $W_M$: Weighted matches (expected = observed)
    \item $W_B$: Weighted betrayals (expected cooperate, observed defect)
    \item $W_U$: Weighted surprises (expected defect, observed cooperate)
    \item $\lambda = 3.0$: Loss aversion parameter (betrayal penalty)
    \item $\mu = 0.5$: Surprise penalty parameter
\end{itemize}

\textbf{Signal Update:} Exponentially weighted moving average with noise:
\begin{equation}
x_{t+1} = x_t + \eta(P_{\text{obs}} - x_t) + \mathcal{N}(0, \sigma^2)
\end{equation}

\subsection{Partner Strategies Tested}
\begin{enumerate}[nosep]
    \item \textbf{Fixed Strategies:} Always Cooperate/Defect, Random, Probabilistic
    \item \textbf{Reactive Strategies:} Tit-for-Tat (Cooperate/Defect start)
    \item \textbf{Temporal Strategies:} Periodic Cheater, Single Cycle
    \item \textbf{Adaptive Strategies:} Strategic Cheater, Expectation Violation
    \item \textbf{Learning Partners:} Belief-driven with own trust/signal dynamics
\end{enumerate}

\section{Key Findings}

\subsection{Trust Formation Patterns}

\textbf{Bayesian Agent:}
\begin{itemize}[nosep]
    \item Rapid convergence to true cooperation rates
    \item Symmetric response to positive/negative evidence
    \item Final posterior means: Always Cooperate (0.986), Always Defect (0.014)
\end{itemize}

\textbf{Dual-State Agent:}
\begin{itemize}[nosep]
    \item Slower trust recovery after betrayal due to loss aversion
    \item Asymmetric updating creates persistent caution
    \item More gradual belief updates with psychological realism
\end{itemize}

\subsection{Decision-Making Analysis}

\textbf{Cooperation Rates by Strategy:}
\begin{center}
\begin{tabular}{lcc}
\toprule
\textbf{Partner Strategy} & \textbf{Bayesian} & \textbf{Dual-State} \\
\midrule
Always Cooperate & 100\% & 98.6\% \\
Tit-for-Tat Coop & 88.6\% & 85.7\% \\
Probabilistic (70\%) & 62.9\% & 60.8\% \\
Strategic Cheater & 75.7\% & 8.7\% \\
Always Defect & 0\% & 0\% \\
\bottomrule
\end{tabular}
\end{center}

\subsection{Payoff Performance}

The dual-state agent shows:
\begin{itemize}[nosep]
    \item \textbf{Higher resilience} against deceptive strategies
    \item \textbf{Comparable performance} with honest partners
    \item \textbf{Better risk management} in uncertain environments
\end{itemize}

\subsection{Loss Aversion Impact}

Sensitivity analysis reveals:
\begin{itemize}[nosep]
    \item $\lambda \in [1.0, 5.0]$ significantly affects trust recovery time
    \item Higher loss aversion → slower trust building but better exploitation resistance
    \item Optimal $\lambda \approx 3.0$ balances trust formation with protection
\end{itemize}

\section{Technical Validation}

\subsection{Model Consistency}
\begin{itemize}[nosep]
    \item Identical partner implementations across models
    \item Same game mechanics and decision thresholds
    \item Consistent random seeds for reproducibility
    \item 70-round simulations with statistical significance testing
\end{itemize}

\subsection{Convergence Analysis}
Both models show stable convergence patterns:
\begin{itemize}[nosep]
    \item Bayesian: Monotonic belief updates
    \item Dual-State: Oscillatory convergence with psychological damping
\end{itemize}

\section{Implications and Applications}

\subsection{Theoretical Contributions}
\begin{enumerate}[nosep]
    \item Novel dual-state architecture for bounded rational trust
    \item Quantification of psychological bias effects in strategic games
    \item Validated benchmark comparison framework
\end{enumerate}

\subsection{Practical Applications}
\begin{itemize}[nosep]
    \item \textbf{Human-AI Collaboration:} Trust-aware system design
    \item \textbf{Multi-Agent Systems:} Robust cooperation protocols
    \item \textbf{Behavioral Economics:} Empirical model validation
    \item \textbf{Security Systems:} Adversary-resistant trust mechanisms
\end{itemize}

\section{Future Research Directions}

\subsection{Model Extensions}
\begin{itemize}[nosep]
    \item Multi-agent trust networks
    \item Dynamic loss aversion parameters
    \item Integration with reputation systems
    \item Real-world experimental validation
\end{itemize}

\subsection{Methodological Improvements}
\begin{itemize}[nosep]
    \item Bayesian parameter estimation from human data
    \item Evolutionary dynamics in agent populations
    \item Cross-cultural trust formation differences
\end{itemize}

\section{Conclusion}

This study demonstrates that incorporating psychological realism into computational trust models significantly alters strategic behavior compared to rational benchmarks. The dual-state agent model provides a tractable framework for analyzing bounded rational trust dynamics while maintaining computational efficiency. Results suggest that loss aversion and asymmetric updating, while reducing short-term cooperation rates, enhance long-term robustness against exploitation.

The validated comparison framework enables rigorous analysis of trust mechanisms in strategic interactions, with immediate applications in human-AI collaboration, multi-agent systems, and behavioral modeling.

\vspace{0.5cm}
\noindent\textbf{Keywords:} Trust dynamics, Stag Hunt game, Bounded rationality, Loss aversion, Multi-agent systems, Behavioral game theory

\vspace{0.5cm}
\noindent\textbf{Code Repository:} Available upon request with complete simulation implementations and analysis notebooks.

\end{document}